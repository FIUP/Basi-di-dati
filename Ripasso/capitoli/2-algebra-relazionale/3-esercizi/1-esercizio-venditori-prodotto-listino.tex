\subsection{Esercizio Venditori, Prodotto, Listino}
\subsubsection{Quesito}

È dato uno schema di basi di dati costituito dalle relazioni: \\

\indent \textbf{VENDITORE}(\underline{vid}, vnome, indirizzo) \\
\indent \textbf{PRODOTTO}(\underline{pid}, pnome, colore, peso) \\
\indent \textbf{LISTINO}(\underline{vid}, \underline{pid}, prezzo) \\

Esistono dei vincoli di integrità referenziale tra vid di VENDITORE
e vid di LISTINO, e tra pid di PRODOTTO e pid di LISTINO. \\

Formulare in algebra relazionale le seguenti interrogazioni:
\begin{enumerate}
  \item Trovare i nomi dei venditori che forniscono prodotti rossi o prodotti verdi
  \item Trovare i nomi dei venditori che hanno a listino almeno due prodotti rossi
  \item Trovare l'id dei venditori che hanno a listino solo prodotti verdi
  \item Trovare l'id dei prodotti a listino più pesanti
\end{enumerate}

\subsubsection{Soluzioni}

\begin{enumerate}
  \item $\Pi_{vnome}(VENDITORE \bowtie LISTINO \bowtie \sigma_{colore="rosso" \lor color="verde"}(PRODOTTO))$
  \item
    S1 := $\sigma_{color="rosso"}(PRODOTTO)$ \\
    S2 := $\Pi_{pid,vid}(LISTINO \bowtie S1)$ \\
    S3 := $\rho_{pid1,vid1 \leftarrow pid,vid}(S2)$ \\
    $\Pi_{vnome}(\sigma_{pid \neq pid1 \land vid = vid1}(S2 \bowtie S3) \bowtie PRODOTTO$
  \item
    S1 := $LISTINO \bowtie \sigma_{colore \neq "verde"} (PRODOTTO))$ \\
    $\Pi_{pid}(LISTINO - S1)$
  \item
    S1 := $\rho_{pid1,pnome1,colore1,peso1 \leftarrow pid,nome,colore,peso}(PRODOTTO)$ \\
    S2 := $\sigma_{peso>peso1}(PRODOTTO \bowtie S1)$
    $\Pi_{pid}(S2)$
\end{enumerate}