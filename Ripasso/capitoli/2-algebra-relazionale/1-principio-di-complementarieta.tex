\subsection{Principio di complementarietà}

Data un'interrogazione $q$ da realizzare in algebra relazionale, spesso la si può
scomporre in sottointerrogazioni che possono corrispondere a diversi idiomi,
frequentemente richiesti in sede d'esame. Molti di questi si basano sul \textbf{Principio di Complementarietà}:

\begin{displaymath}
  \sigma_{p}(R) \equiv R - \sigma_{\neg{p}}(R)
\end{displaymath}

La selezione fatta su una relazione $R$ con un predicato $p$, è uguale alla relazione $R$
stessa meno la selezione su $R$ con predicato $p$ negato.

\begin{center}
  \begin{tabular}{ | l | l | l | c |}
    \hline
      \textbf{Studente} & \textbf{Corso} & \textbf{Data} & \textbf{Voto} \\ \hline
      Luca & Basi di Dati & 23/11/2017 & 30 \\ \hline
      Anna & Logica & 22/11/2017 & 22 \\ \hline
      Marco & Programmazione & 20/09/2017 & 18 \\ \hline
      Anna & Programmazione & 21/10/2017 & 30 \\
    \hline
  \end{tabular}
\end{center}

Per esempio nella relazione ESAMI, l’ insieme degli studenti che ha preso 30 è dato
dall’ insieme di tutti gli studenti che hanno dato almeno un esame (quelli nella tabella
\textbf{ESAMI}) meno tutti gli studenti che non hanno preso 30. Gli idiomi di interrogazione
principali sono:

\begin{enumerate}
  \item Minimo e Massimo (assoluti o relativi)
  \item Cardinalità
  \item Per ogni
  \item Inclusione
\end{enumerate}