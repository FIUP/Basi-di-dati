\subsubsection{Minimo e Massimo assoluto}

Dato lo schema relazionale $R(A,B)$, trovare il minimo/massimo in R.
Si supponga di voler determinare il minimo $B$:

\begin{displaymath}
  \pi_{B}(R) - \pi_{B}(R \bowtie_{B>B^1}(\rho{A^1,B^1 \leftarrow A,B}(R)))
\end{displaymath}

Nella seconda parte vengono trovati tutti quei valori che non sono il minimo.
Per fare ciò si deve joinare la relazione $R$ con un'altra istanza di se stessa,
con gli attributi ridenominati. La condizione del theta join indica che ogni attributo
$B$ deve essere maggiore degli stessi attributi ridenominati. In tal modo vengono
mantenute tutte le tuple tranne quella in cui l'attributo $B$ assume il valore minore. \\

\noindent
Per il Principio di Complementarietà, sottraendo dall'insieme iniziale l'insieme delle
tuple dove $B$ non è il minimo, si ottiene proprio il valore minimo cercato. \\

\noindent
Esempio con la relazione \textbf{ESAMI} e l'attributo Voto:

\begin{gather*}
  S1 := \rho_{Studente^1,Corso^1,Data^1,Voto^1 \leftarrow Studente,Corso,Data,Voto}(\textbf{ESAMI}) \\
  \pi_{Voto}(\textbf{ESAMI}) - \pi_{Voto}(\textbf{ESAMI} \bowtie_{Voto>Voto^1}(S1))
\end{gather*}