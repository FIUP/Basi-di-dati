\subsubsection{Cardinalità}

Dato lo schema relazionale $R(A,B)$, trovare gli $A$ che sono associati ad almeno 2 $B$:

\begin{displaymath}
  \pi_{A}(R \bowtie_{A=A^1 \land B \neq B^1}(\rho{A^1,B^1 \leftarrow A,B}(R)))
\end{displaymath}

Viene ancora fatto un theta join tra la relazione $R$ e se stessa con gli attributi
ridenominati. Il predicato del join consente di mantenere tutte quelle tuple in cui
l'attributo $A$ è uguale e $B$ è diverso. Queste tuple sono proprio tutte le tuple di $B$
associate almeno 2 volte ad ogni elemento di $A$. \\

\noindent
Dato lo schema relazionale $R(A,B)$, trovare gli $A$ che sono associati ad almeno 3 $B$:

\begin{gather*}
  S1 := \rho_{A^1,B^1 \leftarrow A,B}(R) \\
  S2 := \rho_{A^2,B^2 \leftarrow A,B}(R) \\
  S3 := R \times S1 \times S2
  \pi_{A}(\sigma_{A=A^1 \land A=A^2 \land B \neq B^1 \land B \neq B^2 \land B^1 \neq B^2}(S3))
\end{gather*}

Viene fatto il prodotto cartesiano della relazione e dei suoi due duplicati con gli attributi
rinominati. Dalla relazione che otteniamo cosi facendo, vengono selezionate le tuple che
soddisfano il predicato di selezione, ovvero tutte le tuple in cui $A$ è associato ad almeno
3 $B$. I generale per trovare gli $A$ che sono associati ad almeno $n$ $B$ bisogna fare: \\
\begin{center}
  \begin{tabular}{|r|l|}
    \hline
    $(n-1)$ & prodotti cartesiani \\
    $(n-1)$ & condizioni della forma $A=A^i$ \\
    $\frac{n(n-1)}{2}$ & condizioni della forma $B^i \neq B^j$ \\
    \hline
  \end{tabular}
\end{center}