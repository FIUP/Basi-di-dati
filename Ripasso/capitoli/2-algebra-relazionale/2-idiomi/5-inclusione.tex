\subsubsection{Inclusione}

Dati gli schemi relazionali $R(A,B)$ e $S(B)$, trovare gli $A$ per i quali l'insieme
dei $B$ associati include tutti gli elementi di $S$.

\begin{displaymath}
  \pi_{A}(R) - \pi_{A}((\pi_{A}(R) \times S) - R)
\end{displaymath}

Esempio: \\

\noindent
\textbf{ESAME}(Studente, CCorso, Voto)
\textbf{Corso}(CCorso, Docente) \\
\noindent

Determinare gli studenti che hanno passato tutti gli esami:

\begin{gather*}
  S1 := \pi_{Studente,CCorso}(\textbf{STUDENTE}) \\
  S2 := \pi_{CCorso}(\textbf{CORSO}) \\
  \pi_{Studente}(S1) - \pi_{Studente}((\pi_{Studente}(S1) \times S2) - S1)
\end{gather*}